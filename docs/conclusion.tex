\chapter*{Заключение}
\addcontentsline{toc}{chapter}{Заключение}

В результате выполнения курсовой работы был разработан загружаемый модуль ядра Linux, позволяющий перехватывать системные вызовы \code{brk}, \code{mmap} и \code{munmap} и выводить информацию о потреблении процессом памяти.

В процессе выполнения курсовой работы были выполнены все поставленные задачи в полном объеме, а именно: проанализированы стандартный аллокатор памяти, коды ядра и методы перехвата системных вызовов, спроектированы структуры данных и алгоритмы, реализован загружаемый модуль ядра, исследовано распределение памяти в однопоточных и многопоточных приложениях.

Было показано, что потребление памяти не зависит от места выделения памяти малых объемов при работе с одним дочерним потоком. Выделение областей памяти объемом более 128 килобайтов в разных потоках приводит к многократным вызовам \code{mmap} и неэффективному использованию памяти, при этом потребление памяти кучи изменяется незначительно.