\chapter*{Введение}
\addcontentsline{toc}{chapter}{Введение}

В настоящее время с ростом объемов приложений --- кода и объема обрабатываемых данных --- важнейшей задачей является мониторинг потребления процессом ресурсов системы, в частности памяти. Так, например, распределение памяти в многопоточных приложениях Linux может отличаться от распределения памяти в однопоточных аналогах тех же программ. Распределение памяти в многопоточных приложениях может приводить неэффективному использованию памяти и, как следствие, к замедлению работы программы и системы в целом. В связи с этим, помимо стандартного средства выделения памяти в Linux (функции \code{malloc()} низкоуровневой библиотеки glibc), разработчики предлагают использование таких менеджеров памяти, как mimalloc от корпорации Microsoft, tcmalloc от Google, jemalloc, разработанный Джейсоном Эвансом для FreeBSD 7.0.

Курсовая работа посвящена разработке программного обеспечения для перехвата системных вызовов с целью анализа информации об адресном пространстве процесса и исследованию распределения памяти в многопоточных приложения Linux.