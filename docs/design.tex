\chapter{Конструкторский раздел}

\section{Проектирование загружаемого модуля ядра}

При загрузке модуль ядра получает адрес таблицы системных вызовов в шестнадцатеричном виде, конвертирует его в указатель, инициализирует функции-перехватчики для системных вызовов \code{brk}, \code{mmap} и \code{munmap}, модифицируя системную таблицу.

При вызове одной из перехваченных функций выполняется функция журналирования данных о размере областей памяти адресного пространства вызывающего процесс. В системный журнал записывается информация о размерах кучи, стека и анонимных отображений.

При выгрузке модуля выполняется обратная замена адресов в таблице системных вызовов и освобождение памяти, выделенной под хранение необходимых структур данных.

На рисунке \ref{img:structure} представлена структура программного обеспечения.

\imgwc{h}{1\textwidth}{structure}{Cтруктура загружаемого модуля ядра}

\subsection{Алгоритмы перехвата системных вызовов}

Алгоритм встраивания функций-перехватчиков представлен на рисунке \ref{img:init}.

Вызов подпрограммы \code{sys\_hook\_init()} инициализирует структуру данных, описывающую функции-перехватчики. Далее инициализируются структуры данных, описывающие каждый конкретный перехватчик.

\imgwc{h}{0.65\textwidth}{init}{Алгоритм встраивания функций-перехватчиков}

Все функции-перехватчики одинаковы по своей структуре: определяется адрес оригинального системного вызова, выполняется оригинальный вызов, затем вызывается функция журналирования данных. Схема алгоритма функции-перехватчика представлена на рисунке \ref{img:hook}.

\imgwc{h}{0.5\textwidth}{hook}{Алгоритм функции-перехватчика}

\subsection{Алгоритм журналирования данных ядра}

Схема алгоритма журналирования размера областей памяти адресного пространства процесса представлена на рисунке \ref{img:log_vma}.

\imghc{h}{0.95\textheight}{log_vma}{Алгоритм функции журналирования}

\section{Вывод}

В результате проектирования разработаны алгоритмы перехвата системных вызовов и вывода данных о потреблении процессом памяти, что позволяет перейти к реализации загружаемого модуля в программном коде.